\begin{abstract} 
Tree-structured neural networks encode a particular tree geometry for a sentence in the network design. However, these models have at best only slightly outperformed simpler sequence-based models. We hypothesize that neural sequence models like LSTMs are in fact able to discover and implicitly use recursive compositional structure, at least for tasks with clear cues to that structure in the data. We demonstrate this possibility using an artificial data task for which recursive compositional structure is crucial, and find an LSTM-based sequence model can learn to exploit the underlying tree structure. However, its performance consistently lags behind that of tree-structured models, even on large training sets, suggesting that tree-structured models retain an advantage in tasks for which it is necessary to exploit recursive structure.
\end{abstract} 