\documentclass[11pt,a4paper]{article}
\usepackage{acl2015}
\usepackage{times}
\usepackage{latexsym}
% \setlength\titlebox{5cm}    % Expanding the titlebox

%%% Custom additions %%%
% \usepackage{hyperref}
\usepackage{url}
\usepackage[leqno, fleqn]{amsmath}
\usepackage{amssymb}
\usepackage{qtree}
\usepackage{graphicx}
\usepackage{booktabs}
\usepackage{colortbl}
% \usepackage{caption}
\usepackage{subcaption}
\usepackage{xcolor}
\usepackage{color}
\usepackage{tikz}
\usepackage{todonotes}

\newcount\colveccount
\newcommand*\colvec[1]{
        \global\colveccount#1
        \begin{bmatrix}
        \colvecnext
}
\def\colvecnext#1{
        #1
        \global\advance\colveccount-1
        \ifnum\colveccount>0
                \\
                \expandafter\colvecnext
        \else
                \end{bmatrix}
        \fi
}


\newcommand{\nateq}{\equiv}
\newcommand{\natind}{\mathbin{\#}}
%\newcommand{\natneg}{\raisebox{2px}{\tiny\thinspace$\wedge$\thinspace}}
\newcommand{\natneg}{\mathbin{^{\wedge}}}
\newcommand{\natfor}{\sqsubset}
\newcommand{\natrev}{\sqsupset}
\newcommand{\natalt}{\mathbin{|}}
\newcommand{\natcov}{\mathbin{\smallsmile}}

\newcommand{\qq}{\hspace{0.1em}}
\newcommand{\plneg}{\mathop{\textit{not}\qq}}
\newcommand{\pland}{\mathbin{\qq\textit{and}\qq}}
\newcommand{\plor}{\mathbin{\qq\textit{or}\qq}}




% Strikeout
\newlength{\howlong}\newcommand{\strikeout}[1]{\settowidth{\howlong}{#1}#1\unitlength0.5ex%
\begin{picture}(0,0)\put(0,1){\line(-1,0){\howlong\divide\unitlength}}\end{picture}}

\newcommand{\True}{\texttt{T}}
\newcommand{\False}{\texttt{F}}
\usepackage{stmaryrd}
\newcommand{\sem}[1]{\ensuremath{\llbracket#1\rrbracket}}

\newcommand{\mynote}[1]{{\color{blue}#1}}

\newcommand{\tbchecked}[1]{{\color{red}#1}}

\usepackage{gb4e}
\noautomath

\def\ii#1{\textit{#1}}
\newcommand{\word}[1]{\emph{#1}}

%%%%%%%%%%%%%%%%%%%%%%%%%%%%%%%%%%%%%%%%%%%%%%%%%%%%%%%%%%%%%%%%%%%%%%
%%%%% Code to simulate natbib's citealt, which prints citations with
%%%%% no parentheses:

\makeatletter
\def\citealt{\def\citename##1{{\frenchspacing##1} }\@internalcitec}
\def\@citexc[#1]#2{\if@filesw\immediate\write\@auxout{\string\citation{#2}}\fi
  \def\@citea{}\@citealt{\@for\@citeb:=#2\do
    {\@citea\def\@citea{;\penalty\@m\ }\@ifundefined
       {b@\@citeb}{{\bf ?}\@warning
       {Citation `\@citeb' on page \thepage \space undefined}}%
{\csname b@\@citeb\endcsname}}}{#1}}
\def\@internalcitec{\@ifnextchar [{\@tempswatrue\@citexc}{\@tempswafalse\@citexc[]}}
\def\@citealt#1#2{{#1\if@tempswa, #2\fi}}
\makeatother

%%%%%%%%%%%%%%%%%%%%%%%%%%%%%%%%%%%%%%%%%%%%%%%%%%%%%%%%%%%%%%%%%%%%%%


%%% %%%

\title{Tree-structured composition in neural networks\\without tree-structured architectures}

%\Thanks{}}

\author{
Samuel R.\ Bowman$^{\ast\dag}$ \\
\texttt{sbowman@stanford.edu} \\
\And
Christopher D.\ Manning$^{\ast\dag\ddag}$\\
\texttt{manning@stanford.edu}\\
\And
Christopher Potts$^{\ast}$\\
\texttt{cgpotts@stanford.edu}
\AND\\[-3ex]
{$^{\ast}$Stanford Linguistics\quad
$^{\dag}$Stanford NLP Group\quad
$^{\ddag}$Stanford Computer Science}
}


\date{}

\makeatletter
\newcommand{\@BIBLABEL}{\@emptybiblabel}
\newcommand{\@emptybiblabel}[1]{}
\definecolor{black}{rgb}{0,0,0}
\makeatother
\usepackage[breaklinks, draft, colorlinks, linkcolor=black, urlcolor=black, citecolor=black]{hyperref}

\begin{document}
\maketitle

\begin{abstract} 
Understanding entailment and contradiction is  fundamental to understanding natural language, and inference about entailment and contradiction is a valuable testing ground for the development of semantic representations. 
However, research in this area has been dramatically limited by the lack of large-scale resources. 
To address this, we introduce a new freely available corpus of labeled sentence pairs, written by humans in a novel grounded task framing based on image captioning. 
At 570K pairs, it is two orders of magnitude larger than all other resources of its type. 
In this paper, we use this corpus to evaluate a wide variety of natural language inference models, finding that Tree-Structured Long Short-Term Memory networks (TreeLSTMs) achieve the best performance. 
In addition, we show that models trained on our corpus perform competitively on an existing NLI test set.
\end{abstract}

\section{Introduction}\label{sec:intro}

Neural network models that encode sentences as real-valued vectors have been successfully used in a wide array of NLP tasks, including translation \cite{sutskever2014sequence}, parsing \cite{dyer2015transition}, and sentiment analysis \cite{tai2015improved}. 
% Many of the most successful of these models are based on either tree or sequence architectures. 
These models for language may be either 
sequence models based on recurrent neural networks, which build representations incrementally from left to right \cite{elman1990finding,sutskever2014sequence}, or tree models based on \ii{recursive} neural networks \cite{goller1996learning,socher2011semi}, which build representations incrementally according the hierarchical structure of linguistic phrases. 


While both model classes perform %objectively 
well on many tasks, and both are under active development,
tree models are often presented as the more principled choice, since they align with standard linguistic assumptions about constituent structure and the compositional derivation of complex meanings.
Nevertheless,
tree models have not shown the kinds of dramatic performance improvements over sequence models that their billing would lead one to expect: head-to-head comparisons with sequence models show either modest improvements \cite{tai2015improved} or none at all \cite{li2015tree}. 

We propose a possible explanation for these results: standard sequence models can learn to
%  implicitly 
exploit recursive syntactic structure in generating sentence meaning representations, thereby 
% demonstrating 
learning to use the 
% behavior 
structure that tree-structured models are explicitly designed around. This first requires that
%This would require both that 
sequence models are able to identify syntactic structure in natural language. We believe this is plausible, on the basis of other recent research \cite{Karpathy2015vaurn}. %the way in which they interpret sentences. 
In this paper, we  evaluate whether LSTM models are able to use that structure to guide interpretation, 
focusing on cases where the relevant syntactic structure is clearly indicated in the data.

We compare standard tree and sequence models on their handling of recursive structure by training the models on sentences whose length and recursion depth are limited, and then testing them on longer and more complex sentences, such that only models that exploit the recursive structure will be able to generalize in a way that yields correct interpretations for these test sentences. Our methods are based on those of \newcite{Bowman:Potts:Manning:2014}, who describe an experiment and corresponding artificial dataset which tests this ability in two tree models. We adapt that experiment to sequence models by decorating the statements with an explicit bracketing, and we use this design to compare an LSTM sequence model with three tree models, with a focus on what data each model needs in order to learn the needed generalizations.

As in Bowman et al., we find that standard tree neural networks are able to make the necessary generalizations, with their performance decaying gradually as the structures in the test set grow in size. We also find that extending the training set to include larger structures mitigates this decay. In addition, we find that a sequence model centered on a single-layer LSTM is also able to generalize to unseen large structures, but that it does this only when trained on a larger and more complex training set than is needed by the tree models. 

%\paragraph{Related work}
%Recent successes on parsing by 
Our results engage with those of \newcite{vinyals2014grammar} and \newcite{dyer2015transition}, who find that sequence models can learn to recognize syntactic structure in natural language, at least when trained on explicitly syntactic tasks. The simplest model presented in Vinyals et al.~uses an LSTM sequence model to encode each sentence as a vector, and then generates a linearized parse (a sequence of brackets and constituent labels) with high accuracy using only the information present in the vector. This shows that the LSTM was able to identify the correct syntactic structures and also hints that it was able to develop a generalizable method for encoding these structures in vectors. However, the massive size of the data set needed to train the model---250M tokens---leaves open the possibility that it primarily learned to generate only tree structures that it had already seen, representing them as simple hashes rather than structured objects. If the model were able to map unseen sentences to familiar structures using these hashed representations, it could plausibly have achieved the strong results shown in that paper without actually representing recursive structure, at least in a way that would generalize to substantially new structures. Our experiments show that LSTMs can manipulate tree-structured data, suggesting that there are are no fundamental obstacles to this kind of generalization.
\section{Recursive structure in artificial data}\label{sec:recursion}
\paragraph{Reasoning about entailment} 
The data that we use define a version of the \emph{recognizing textual entailment} task, in which the goal is to determine what kind of logical consequence relation holds between two sentences, drawing on a small fixed vocabulary of relations such as entailment, contradiction, and synonymy. This task is well suited to evaluating neural network models for sentence interpretation: models must develop comprehensive representations of the meanings of each sentence to do well at the task, but the data do not force these representations to take a specific form, allowing the model to learn whatever kind of representations it can use most effectively.

The data we use are labeled with the seven mutually exclusive logical relations of \newcite{maccartney2009extended}, which distinguish entailment in two directions ($\natfor$, $\natrev$), equivalence ($\nateq$), exhaustive and non-exhaustive contradiction ($\natneg$, $\natalt$), and two types of semantic independence ($\natind$, $\natcov$).

%\begin{table}[tp]
%  \centering\small
%  \renewcommand{\arraystretch}{1}
%  \begin{tabular}{l c l l} 
%    \toprule
%    Name & Symb. & Set-theoretic definition \\ 
%    \midrule
%strict entailment         & $\natfor$   & $x \subset y$  \\ 
%    strict rev. entailment & $\natrev$   & $x \supset y$  \\ 
%    equivalence        & $\nateq$    & $x = y$   \\ 
%    alternation        & $\natalt$   & $x \cap y = \emptyset \wedge x \cup y \neq \mathcal{D}$ \\ 
%    negation           & $\natneg$   & $x \cap y = \emptyset \wedge x \cup y = \mathcal{D}$   \\
%    cover              & $\natcov$   & $x \cap y \neq \emptyset \wedge x \cup y = \mathcal{D}$ \\ 
%    independence       & $\natind$   & (else)\\
%    \bottomrule
%  \end{tabular}
%  \caption{\label{b-table}MacCartney's seven mutually exclusive relations are defined abstractly on pairs of sets drawing from the universe $\mathcal{D}$, but can be straightforwardly applied to any pair of natural language words, phrases, or sentences.
%  } %-%
%\end{table}

\begin{table}[tp]
  \centering\small
%  \begin{subtable}[t]{0.45\textwidth}
%    \centering
%    \begin{tabular}[t]{l l}
%      \toprule
%      Formula     & Interpretation \\
%      \midrule
%      $p_1$, $p_2$, $p_3$, $p_4$, $p_5$, $p_6$ & $\sem{x} \in \{\True, \False\}$ \\
%      $\plneg \varphi$ & $\True$ iff $\sem{\varphi} = \False$ \\
%      $(\varphi \pland \psi)$ & $\True$ iff $\False \notin \{\sem{\varphi}, \sem{\psi}\}$ \\
%      $(\varphi \plor \psi)$  & $\True$ iff $\True \in \{\sem{\varphi}, \sem{\psi}\}$ \\
%      \bottomrule
%    \end{tabular}    
%    \caption{Well-formed formulae. $\varphi$ and $\psi$
%      range over all well-formed formulae, and $\sem{\cdot}$ is
%      the interpretation function mapping formulae into $\{\True,
%      \False\}$.}\label{tab:pl}
%  \end{subtable}
    \begin{tabular}[t]{r c l}
      \toprule
      $\plneg p_3$        & $\natneg$ & $p_3$ \\
%      $\plneg \plneg p_6$ & $\nateq$  & $p_6$ \\
      $p_3$               & $\natfor$ & $p_3 \plor p_2$ \\
%      $p_1 \plor (p_2 \plor p_4)$               & $\natrev$ & $p_2 \pland  \plneg p_4$ \\
      $(\plneg p_2) \pland p_6  $ & $\natalt$ & $     \plneg ( p_6 \plor ( p_5 \plor p_3 ) ) $\\
      %$(a \natfor b)$   & $\nateq$  & $(b \natrev a)$ \\	
 $    p_4 \plor ( \plneg ( ( p_1   $ & $ \natfor $ & $  \plneg ( ( ( ( \plneg p_6 ) \plor ( \plneg p_4 ) )   $\\      
$ \plor p_6 )  \plor p_4 ) )$ &&$ \pland ( \plneg p_5 ) )\pland ( p_6 \pland p_6 ) ) $\\
% $(p_3 \pland \plneg p_1 ) \plor \plneg p_3$    & $\natrev$& $\plneg\, (p_3 \plor p_2)$ \\
      %<	( not ( c ( or b ) ) )	( ( c ( and ( not a ) ) ) ( or ( not c ) ) )
      \bottomrule
    \end{tabular}
    \caption{Examples of short to moderate length pairs from the artificial data introduced in \protect\citealt{Bowman:Potts:Manning:2014}. We only show the parentheses that are needed to disambiguate the sentences rather than the full binary bracketings.}\label{tab:plexs}
\end{table}


\paragraph{The artificial language} The language described in Bowman et al.~(\S4) is designed to highlight the use of recursive structure with minimal additional complexity. Its vocabulary consists only of six unanalyzed word types ($p_1, p_2, p_3, p_4, p_5, p_6$), \word{and}, \word{or}, and \word{not}. Sentences of the language can be straightforwardly interpreted as statements of propositional logic (where the six unanalyzed words are variables), and labeled sentence pairs can be interpreted as theorems of that logic. Some example pairs are provided in Table~\ref{tab:plexs}.

Crucially, the language is defined such that any sentence can be embedded under negation or conjunction to create a new sentence, allowing for arbitrary-depth recursion, and the scope of negation and conjunction are determined only by bracketing with parentheses (rather than bare word order). The compositional structure of each sentence can thus be an arbitrary tree, and interpreting a sentence correctly requires using that structure.

The data come with parentheses representing a complete binary bracketing. Our models use this information in two ways. For the tree models, the parentheses are not word tokens, but rather used in the expected way to build the tree. For the sequence model, the parentheses are words with associated learned embeddings. This provides the models with equivalent data, so their ability to handle unseen structures can be fairly compared.

\paragraph{The corpus}
The sentence pairs in the corpus are divided into thirteen bins according to the number of logical connectives (\word{and, or, not}) in the longer of the two sentences in the pair. We test the model on each bin separately (58k total examples, using an 80/20\% train/test split) in order to evaluate how its performance depends on the complexity of the sentences. In three experiments, we train our models on the training portions of bins 0--3 (62k examples), 0--4 (90k), and 0--6 (160k), and test on every bin but the trivial bin 0. Capping the size of the training sentences allows us to evaluate how the models interpret the sentences: if their performance falls off abruptly above the cutoff, it is reasonable to assume that the models are depending heavily on specific sentence structures, and cannot generalize to new structures. If their performance decays gradually\footnote{Since sentences are fixed-dimensional vectors of fixed-precision floating point numbers, all models will make errors on sentences above some length, and L2 regularization (which helps overall performance) exacerbates this by discouraging the model from using the kind of numerically precise, nonlinearity-saturating functions that generalize best.} with no such abrupt change, then it must have learned a more generally valid interpretation function for the language which respects its recursive structure.





\subsection*{Neural network models for relation classification} \label{methods}

% TODO: Should we cite our NIPS manuscript?

We follow the approach to learning semantically meaningful embeddings 
proposed in \citet{bowman2013can}, which is centered on the problem of
labeling a pair of words or sentences with one of a small set of logical
relations. The architecture of the model that we use, which is limited
to only pairs of single symbols (such as words), is depicted in
Figure~\ref{sample-figure}. The model represents the two input symbols
as embeddings, which are fed into a comparison function based on one
of two types of neural network layer functions to produce a representation
for the relationship between the two symbols. This representation is then
fed into a simple softmax classifier which outputs a distribution over
possible labels. The entire network, including the embeddings, are trained
through backpropagation with AdaGrad \cite{duchi2011adaptive}.

\begin{figure}[tp]
  \centering
  \footnotesize

\newcommand{\labeledtreenode}[4][3.5]{\put(#2){\makebox(0,0){{\fcolorbox{black}{#4}{\makebox(#1,0.3){#3}}}}}}

\newcommand{\textlabel}[4][3.5]{\put(#2){\makebox(0,0){{\fcolorbox{white}{white}{\makebox(#1,0.3){#3}}}}}}

\definecolor{lexcolor}{HTML}{F5F7C4}
\definecolor{compositioncolor}{HTML}{BBEBFF}
\definecolor{comparisoncolor}{HTML}{FFC895}
\definecolor{softmaxcolor}{HTML}{A5FF8A}


\setlength{\unitlength}{0.61cm}
\begin{picture}(21,7.5)
  
  \labeledtreenode[2.4]{11.5,7}{$P(\sqsubset) = 0.8$}{softmaxcolor}  
  \put(11.5,5.7){\vector(0,1){1}}  
  \labeledtreenode[7.85]{11.5,5.4}{all reptiles walk \emph{vs.}~some turtles move}{comparisoncolor}


  \textlabel{8,7}{Softmax classifier}{black}
  \textlabel{4.5,5.4}{Comparison N(T)N layer}{black}
      
  \textlabel{11.75,3.6}{Composition RN(T)N layers}{black}

  \textlabel{5,0.1}{Learned, randomly initialized word vectors}{black}
  
  %%%%%%%%%%%%%%%%%%%%%%%%%%%%%%%%%%%%%%%%%%%%%%%%%%
    
  \put(1.75,1.35){\vector(2,1){1.7}}
  \labeledtreenode{1.75,1}{all}{lexcolor}

  \put(6,1.35){\vector(-2,1){1.7}}
  \labeledtreenode{6,1}{reptiles}{lexcolor}

  \put(4,2.75){\vector(2,1){1.7}}
  \labeledtreenode{4,2.5}{all reptiles}{compositioncolor}

  \put(8.25,2.75){\vector(-2,1){1.7}}
  \labeledtreenode{8.25,2.5}{walk}{lexcolor}

  \put(6.25,4.25){\vector(4,1){3.25}}
  \labeledtreenode{6.25,3.9}{all reptiles walk}{compositioncolor}
  
  %%%%%%%%%%%%%%%%%%%%%%%%%%%%%%%%%%%%%%%%%%%%%%%%%%%

  \put(12.75,1.35){\vector(2,1){1.7}}
  \labeledtreenode{12.75,1}{some}{lexcolor}

  \put(17,1.35){\vector(-2,1){1.7}}
  \labeledtreenode{17,1}{turtles}{lexcolor}

  \put(15,2.75){\vector(2,1){1.7}}
  \labeledtreenode{15,2.5}{some turtles}{compositioncolor}

  \put(19.25,2.75){\vector(-2,1){1.7}}
  \labeledtreenode{19.25,2.5}{move}{lexcolor}
          
  \put(17.25,4.25){\vector(-4,1){3.25}}
  \labeledtreenode{17.25,3.9}{some turtles move}{compositioncolor}
  
\end{picture}



  \caption{The model structure used to compare \ii{turtle} and \ii{animal}. 
    The same structure is used for both the RNN and RNTN layer functions.} 
  \label{sample-figure}
\end{figure}

For a comparison funtion, we evaluate versions of the model with both a plain neural
network (NN) layer function and a neural tensor network (NTN) layer function
\eqref{rntn} proposed in \citet{chen2013learning}. A leaky ReLU
nonlinearity \cite{maasrectifier} is applied to the output of either
layer function.
%
\begin{gather} \label{rnn}
\vec{y}_{\textit{NN}} = f(\mathbf{M} [\vec{x}^{(l)}; \vec{x}^{(r)}] + \vec{b}) \\ % TODO: Add column vectors?
\label{rntn}
\vec{y}_{\textit{NTN}} = f(\vec{x}^{(l)T} \mathbf{T}^{[1 \ldots n]} \vec{x}^{(r)} + \mathbf{M} [\vec{x}^{(l)}; \vec{x}^{(r)}] + \vec{b})
\end{gather} 
%
Here, $\vec{x}^{(l)}$ and $\vec{x}^{(r)}$ are the column vector
representations for the left and right children of the node, and
$\vec{y}$ is the node's output.  The RNN concatenates them, multiplies
them by an $n \times 2n$ matrix of learned weights, and applies the
element-wise non-linearity to the resulting vector. The RNTN has the
same basic structure, but with the addition of a learned third-order
tensor $\mathbf{T}$, dimension $n \times n \times n$, modeling
multiplicative interactions between the child vectors. Both models
include a bias vector~$\vec{b}$.


This model differs from that of \citet{bowman2013can} in two ways. Because 
the inputs are single symbols, there is no need for the composition functions
which are used in that (and prior) work. Also, for our second experiment on 
WordNet data, we introduce a new neural network layer between the embedding input
and the comparison function, which is meant to facilitate initializing the embeddings
from an outside source, and was found to help peformance in that setting.

%\ii{Source code and generated data will be released after the conclusion of the review period.} % TODO: Or upon request? Attach anonymized code?


\section{Results and discussion}\label{sec:discussion}

The results are shown in Figure \ref{prop-results}. All three models were able to largely memorize the training data (TODO, TODO, and TODO for $\le$6). With sufficient training data (in the $\le$6 setting), the two tree models were able to perform at above 99\% (TODO) on examples of size $\le$2 with a steady decay in performance continuing through TODO\% at size 12, and similar, if steeper, decays with smaller training sets. These robustly beat the simple baselines reported in Bowman et al.: the most frequent class occurs just over 50\% of the time, and a summing model that ignores word order does reasonably on the shortest examples but falls below 60\% by bin 4.

The LSTM performs fairly poorly in the $\le3$ setting, with performance at 3 a mediorce TODO\%, and an abrubt drop to TODO\% at 4, suggesting that the model did not acquire the needed generalizations. With more ample training data in the $\le6$ condition, though, the LSTM shows fairly good accuracy on short examples, and a smooth decay with no abrupt drop. 

\paragraph{Conclusion}

We find that all three \todo{Still using three?} models can exploit recursive structure to interpret sentences with complex unseen structures, and that tree structured model's biases allow them to learn to do this more effectively from less data. 
We interpret these results as evidence that the similar performance of tree and sequence models can be attributed to their fundamentally similar representational systems, with both models recognizing compositional structure when it is present, the former doing so more reliably, and the latter benefiting elsewhere---either because of architectural choices that make backpropagation more effective, or because interpreting sentences from left to right allows the model to better simulate other aspects of human language comprehension. We finally suggest that there is value in attempting to further develop this ability in sequence models without fully sacrificing the flexibility that makes these models succeed.


%\subsubsection*{Acknowledgments}

% We thank Jeffrey Pennington, Richard Socher, and audiences at CSLI, Nuance, and BayLearn, as well as Neha Nayak for developing the SICK collapsing technique.

\bibliographystyle{acl}
\bibliography{MLSemantics} 

\end{document}
