%
% File naaclhlt2015.tex
%

\documentclass[11pt]{article}
\usepackage{acl2015}
\usepackage{times}
\usepackage{latexsym}
% \setlength\titlebox{5cm}    % Expanding the titlebox

%%% Custom additions %%%
% \usepackage{hyperref}
\usepackage{url}
\usepackage[leqno, fleqn]{amsmath}
\usepackage{amssymb}
\usepackage{qtree}
\usepackage{graphicx}
\usepackage{booktabs}
\usepackage{colortbl}
% \usepackage{caption}
\usepackage{subcaption}
\usepackage{xcolor}
\usepackage{color}


\newcount\colveccount
\newcommand*\colvec[1]{
        \global\colveccount#1
        \begin{bmatrix}
        \colvecnext
}
\def\colvecnext#1{
        #1
        \global\advance\colveccount-1
        \ifnum\colveccount>0
                \\
                \expandafter\colvecnext
        \else
                \end{bmatrix}
        \fi
}


\newcommand{\nateq}{\equiv}
\newcommand{\natind}{\mathbin{\#}}
%\newcommand{\natneg}{\raisebox{2px}{\tiny\thinspace$\wedge$\thinspace}}
\newcommand{\natneg}{\mathbin{^{\wedge}}}
\newcommand{\natfor}{\sqsubset}
\newcommand{\natrev}{\sqsupset}
\newcommand{\natalt}{\mathbin{|}}
\newcommand{\natcov}{\mathbin{\smallsmile}}

\newcommand{\plneg}{\mathop{\textit{not}}}
\newcommand{\pland}{\mathbin{\textit{and}}}
\newcommand{\plor}{\mathbin{\textit{or}}}

% Strikeout
\newlength{\howlong}\newcommand{\strikeout}[1]{\settowidth{\howlong}{#1}#1\unitlength0.5ex%
\begin{picture}(0,0)\put(0,1){\line(-1,0){\howlong\divide\unitlength}}\end{picture}}

\newcommand{\True}{\texttt{T}}
\newcommand{\False}{\texttt{F}}
\usepackage{stmaryrd}
\newcommand{\sem}[1]{\ensuremath{\llbracket#1\rrbracket}}

\newcommand{\mynote}[1]{{\color{blue}#1}}

\newcommand{\tbchecked}[1]{{\color{red}#1}}

\usepackage{gb4e}
\noautomath

\def\ii#1{\textit{#1}}
\newcommand{\word}[1]{\emph{#1}}

%%%%%%%%%%%%%%%%%%%%%%%%%%%%%%%%%%%%%%%%%%%%%%%%%%%%%%%%%%%%%%%%%%%%%%
%%%%% Code to simulate natbib's citealt, which prints citations with
%%%%% no parentheses:

\makeatletter
\def\citealt{\def\citename##1{{\frenchspacing##1} }\@internalcitec}
\def\@citexc[#1]#2{\if@filesw\immediate\write\@auxout{\string\citation{#2}}\fi
  \def\@citea{}\@citealt{\@for\@citeb:=#2\do
    {\@citea\def\@citea{;\penalty\@m\ }\@ifundefined
       {b@\@citeb}{{\bf ?}\@warning
       {Citation `\@citeb' on page \thepage \space undefined}}%
{\csname b@\@citeb\endcsname}}}{#1}}
\def\@internalcitec{\@ifnextchar [{\@tempswatrue\@citexc}{\@tempswafalse\@citexc[]}}
\def\@citealt#1#2{{#1\if@tempswa, #2\fi}}
\makeatother

%%%%%%%%%%%%%%%%%%%%%%%%%%%%%%%%%%%%%%%%%%%%%%%%%%%%%%%%%%%%%%%%%%%%%%


%%% %%%

\title{Recursive Neural Networks Can Learn Logical Semantics}

%\Thanks{}}

\author{
Samuel R.\ Bowman$^{\ast\dag}$ \\
\texttt{sbowman@stanford.edu} \\[2ex]
\And
Christopher Potts$^{\ast}$\\
\texttt{cgpotts@stanford.edu} \\[2ex]
\{$^{\ast}$Dept. of Linguistics, $^{\dag}$NLP Group, $^{\ddag}$Dept. of Computer Science\}\\
Stanford University \\
Stanford, CA 94305, USA
\And
Christopher D.\ Manning$^{\ast\dag\ddag}$\\
\texttt{manning@stanford.edu}\\[2ex]
}

%\author{First Author \\
%  Affiliation / Address line 1 \\
%  Affiliation / Address line 2 \\
%  Affiliation / Address line 3 \\
%  {\tt email@domain} \\\And
%  Second Author \\
%  Affiliation / Address line 1 \\
%  Affiliation / Address line 2 \\
%  Affiliation / Address line 3 \\
%  {\tt email@domain} \\}

\date{}

\makeatletter
\newcommand{\@BIBLABEL}{\@emptybiblabel}
\newcommand{\@emptybiblabel}[1]{}
\definecolor{black}{rgb}{0,0,0}
\makeatother
\usepackage[breaklinks, draft, colorlinks, linkcolor=black, urlcolor=black, citecolor=black]{hyperref}

\begin{document}
\maketitle

\begin{abstract} 
Understanding entailment and contradiction is  fundamental to understanding natural language, and inference about entailment and contradiction is a valuable testing ground for the development of semantic representations. 
However, research in this area has been dramatically limited by the lack of large-scale resources. 
To address this, we introduce a new freely available corpus of labeled sentence pairs, written by humans in a novel grounded task framing based on image captioning. 
At 570K pairs, it is two orders of magnitude larger than all other resources of its type. 
In this paper, we use this corpus to evaluate a wide variety of natural language inference models, finding that Tree-Structured Long Short-Term Memory networks (TreeLSTMs) achieve the best performance. 
In addition, we show that models trained on our corpus perform competitively on an existing NLI test set.
\end{abstract}

\section{Introduction}\label{sec:intro}

Neural network models that encode sentences as real-valued vectors have been successfully used in a wide array of NLP tasks, including translation \cite{sutskever2014sequence}, parsing \cite{dyer2015transition}, and sentiment analysis \cite{tai2015improved}. 
% Many of the most successful of these models are based on either tree or sequence architectures. 
These models for language may be either 
sequence models based on recurrent neural networks, which build representations incrementally from left to right \cite{elman1990finding,sutskever2014sequence}, or tree models based on \ii{recursive} neural networks \cite{goller1996learning,socher2011semi}, which build representations incrementally according the hierarchical structure of linguistic phrases. 


While both model classes perform %objectively 
well on many tasks, and both are under active development,
tree models are often presented as the more principled choice, since they align with standard linguistic assumptions about constituent structure and the compositional derivation of complex meanings.
Nevertheless,
tree models have not shown the kinds of dramatic performance improvements over sequence models that their billing would lead one to expect: head-to-head comparisons with sequence models show either modest improvements \cite{tai2015improved} or none at all \cite{li2015tree}. 

We propose a possible explanation for these results: standard sequence models can learn to
%  implicitly 
exploit recursive syntactic structure in generating sentence meaning representations, thereby 
% demonstrating 
learning to use the 
% behavior 
structure that tree-structured models are explicitly designed around. This first requires that
%This would require both that 
sequence models are able to identify syntactic structure in natural language. We believe this is plausible, on the basis of other recent research \cite{Karpathy2015vaurn}. %the way in which they interpret sentences. 
In this paper, we  evaluate whether LSTM models are able to use that structure to guide interpretation, 
focusing on cases where the relevant syntactic structure is clearly indicated in the data.

We compare standard tree and sequence models on their handling of recursive structure by training the models on sentences whose length and recursion depth are limited, and then testing them on longer and more complex sentences, such that only models that exploit the recursive structure will be able to generalize in a way that yields correct interpretations for these test sentences. Our methods are based on those of \newcite{Bowman:Potts:Manning:2014}, who describe an experiment and corresponding artificial dataset which tests this ability in two tree models. We adapt that experiment to sequence models by decorating the statements with an explicit bracketing, and we use this design to compare an LSTM sequence model with three tree models, with a focus on what data each model needs in order to learn the needed generalizations.

As in Bowman et al., we find that standard tree neural networks are able to make the necessary generalizations, with their performance decaying gradually as the structures in the test set grow in size. We also find that extending the training set to include larger structures mitigates this decay. In addition, we find that a sequence model centered on a single-layer LSTM is also able to generalize to unseen large structures, but that it does this only when trained on a larger and more complex training set than is needed by the tree models. 

%\paragraph{Related work}
%Recent successes on parsing by 
Our results engage with those of \newcite{vinyals2014grammar} and \newcite{dyer2015transition}, who find that sequence models can learn to recognize syntactic structure in natural language, at least when trained on explicitly syntactic tasks. The simplest model presented in Vinyals et al.~uses an LSTM sequence model to encode each sentence as a vector, and then generates a linearized parse (a sequence of brackets and constituent labels) with high accuracy using only the information present in the vector. This shows that the LSTM was able to identify the correct syntactic structures and also hints that it was able to develop a generalizable method for encoding these structures in vectors. However, the massive size of the data set needed to train the model---250M tokens---leaves open the possibility that it primarily learned to generate only tree structures that it had already seen, representing them as simple hashes rather than structured objects. If the model were able to map unseen sentences to familiar structures using these hashed representations, it could plausibly have achieved the strong results shown in that paper without actually representing recursive structure, at least in a way that would generalize to substantially new structures. Our experiments show that LSTMs can manipulate tree-structured data, suggesting that there are are no fundamental obstacles to this kind of generalization.

\subsection*{Neural network models for relation classification} \label{methods}

% TODO: Should we cite our NIPS manuscript?

We follow the approach to learning semantically meaningful embeddings 
proposed in \citet{bowman2013can}, which is centered on the problem of
labeling a pair of words or sentences with one of a small set of logical
relations. The architecture of the model that we use, which is limited
to only pairs of single symbols (such as words), is depicted in
Figure~\ref{sample-figure}. The model represents the two input symbols
as embeddings, which are fed into a comparison function based on one
of two types of neural network layer functions to produce a representation
for the relationship between the two symbols. This representation is then
fed into a simple softmax classifier which outputs a distribution over
possible labels. The entire network, including the embeddings, are trained
through backpropagation with AdaGrad \cite{duchi2011adaptive}.

\begin{figure}[tp]
  \centering
  \footnotesize

\newcommand{\labeledtreenode}[4][3.5]{\put(#2){\makebox(0,0){{\fcolorbox{black}{#4}{\makebox(#1,0.3){#3}}}}}}

\newcommand{\textlabel}[4][3.5]{\put(#2){\makebox(0,0){{\fcolorbox{white}{white}{\makebox(#1,0.3){#3}}}}}}

\definecolor{lexcolor}{HTML}{F5F7C4}
\definecolor{compositioncolor}{HTML}{BBEBFF}
\definecolor{comparisoncolor}{HTML}{FFC895}
\definecolor{softmaxcolor}{HTML}{A5FF8A}


\setlength{\unitlength}{0.61cm}
\begin{picture}(21,7.5)
  
  \labeledtreenode[2.4]{11.5,7}{$P(\sqsubset) = 0.8$}{softmaxcolor}  
  \put(11.5,5.7){\vector(0,1){1}}  
  \labeledtreenode[7.85]{11.5,5.4}{all reptiles walk \emph{vs.}~some turtles move}{comparisoncolor}


  \textlabel{8,7}{Softmax classifier}{black}
  \textlabel{4.5,5.4}{Comparison N(T)N layer}{black}
      
  \textlabel{11.75,3.6}{Composition RN(T)N layers}{black}

  \textlabel{5,0.1}{Learned, randomly initialized word vectors}{black}
  
  %%%%%%%%%%%%%%%%%%%%%%%%%%%%%%%%%%%%%%%%%%%%%%%%%%
    
  \put(1.75,1.35){\vector(2,1){1.7}}
  \labeledtreenode{1.75,1}{all}{lexcolor}

  \put(6,1.35){\vector(-2,1){1.7}}
  \labeledtreenode{6,1}{reptiles}{lexcolor}

  \put(4,2.75){\vector(2,1){1.7}}
  \labeledtreenode{4,2.5}{all reptiles}{compositioncolor}

  \put(8.25,2.75){\vector(-2,1){1.7}}
  \labeledtreenode{8.25,2.5}{walk}{lexcolor}

  \put(6.25,4.25){\vector(4,1){3.25}}
  \labeledtreenode{6.25,3.9}{all reptiles walk}{compositioncolor}
  
  %%%%%%%%%%%%%%%%%%%%%%%%%%%%%%%%%%%%%%%%%%%%%%%%%%%

  \put(12.75,1.35){\vector(2,1){1.7}}
  \labeledtreenode{12.75,1}{some}{lexcolor}

  \put(17,1.35){\vector(-2,1){1.7}}
  \labeledtreenode{17,1}{turtles}{lexcolor}

  \put(15,2.75){\vector(2,1){1.7}}
  \labeledtreenode{15,2.5}{some turtles}{compositioncolor}

  \put(19.25,2.75){\vector(-2,1){1.7}}
  \labeledtreenode{19.25,2.5}{move}{lexcolor}
          
  \put(17.25,4.25){\vector(-4,1){3.25}}
  \labeledtreenode{17.25,3.9}{some turtles move}{compositioncolor}
  
\end{picture}



  \caption{The model structure used to compare \ii{turtle} and \ii{animal}. 
    The same structure is used for both the RNN and RNTN layer functions.} 
  \label{sample-figure}
\end{figure}

For a comparison funtion, we evaluate versions of the model with both a plain neural
network (NN) layer function and a neural tensor network (NTN) layer function
\eqref{rntn} proposed in \citet{chen2013learning}. A leaky ReLU
nonlinearity \cite{maasrectifier} is applied to the output of either
layer function.
%
\begin{gather} \label{rnn}
\vec{y}_{\textit{NN}} = f(\mathbf{M} [\vec{x}^{(l)}; \vec{x}^{(r)}] + \vec{b}) \\ % TODO: Add column vectors?
\label{rntn}
\vec{y}_{\textit{NTN}} = f(\vec{x}^{(l)T} \mathbf{T}^{[1 \ldots n]} \vec{x}^{(r)} + \mathbf{M} [\vec{x}^{(l)}; \vec{x}^{(r)}] + \vec{b})
\end{gather} 
%
Here, $\vec{x}^{(l)}$ and $\vec{x}^{(r)}$ are the column vector
representations for the left and right children of the node, and
$\vec{y}$ is the node's output.  The RNN concatenates them, multiplies
them by an $n \times 2n$ matrix of learned weights, and applies the
element-wise non-linearity to the resulting vector. The RNTN has the
same basic structure, but with the addition of a learned third-order
tensor $\mathbf{T}$, dimension $n \times n \times n$, modeling
multiplicative interactions between the child vectors. Both models
include a bias vector~$\vec{b}$.


This model differs from that of \citet{bowman2013can} in two ways. Because 
the inputs are single symbols, there is no need for the composition functions
which are used in that (and prior) work. Also, for our second experiment on 
WordNet data, we introduce a new neural network layer between the embedding input
and the comparison function, which is meant to facilitate initializing the embeddings
from an outside source, and was found to help peformance in that setting.

%\ii{Source code and generated data will be released after the conclusion of the review period.} % TODO: Or upon request? Attach anonymized code?



\begin{table}[tp]
  \centering  \small
  \setlength{\arraycolsep}{8pt}
  \renewcommand{\arraystretch}{1.1}
  \newcommand{\UNK}{\cdot}  
  $\begin{array}[t]{c@{ \ }|*{7}{c}|}
    %\hline
    \multicolumn{1}{c}{}
             & \nateq     & \natfor     & \natrev     & \natneg    & \natalt     & \natcov     & \multicolumn{1}{c}{\natind} \\
    \cline{2-8}
    \nateq  & \nateq &   \natfor &  \natrev &  \natneg &   \natalt &  \natcov &  \natind \\
    \natfor & \natfor &  \natfor &  \UNK &  \natalt &   \natalt &  \UNK &  \UNK \\
    \natrev & \natrev &  \UNK &  \natrev &  \natcov &   \UNK &  \natcov &  \UNK \\
    \natneg & \natneg &  \natcov &  \natalt &  \nateq &    \natrev &  \natfor &  \natind \\
    \natalt & \natalt &  \UNK &  \natalt &  \natfor &   \UNK &  \natfor &  \UNK \\
    \natcov & \natcov &  \natcov &  \UNK &  \natrev &   \natrev &  \UNK &  \UNK \\
    \natind & \natind & \UNK &  \UNK &  \natind &  \UNK &  \UNK &  \UNK \\
    \cline{2-8}
  \end{array}$
  \caption{In \S\ref{sec:join}, we assess our models' ability to learn to do inference over pairs of labels using the rules represented here, which are derived from the definitions of the labels in Table~\ref{b-table}.  As an example, given that $p_1 \natfor p_2$ and $p_2 \natneg p_3$, the entry in the $\natfor$ row and the $\natneg$ column lets us conclude that $p_1 \natalt p_3$. Cells containing a dot correspond to situations for which no valid inference can be drawn.} 
  \label{tab:jointable}
\end{table}

\section{Reasoning about semantic labels}\label{sec:join}

The simplest kinds of deduction in natural logic involve atomic statements 
using the labels in Table~\ref{b-table}. 
For instance, from the label $p_1 \natrev p_2$ between two propositions, 
one can infer the label $p_2 \natfor p_1$ by applying the definitions of the labels directly. 
If one is also given the label $p_2 \natrev p_3$ one can conclude that $p_1 \natrev p_3$, by basic set-theoretic reasoning (transitivity of $\natrev$). The
full set of sound such inferences on pairs of premise labels is depicted in
Table~\ref{tab:jointable}. Though these basic inferences do not involve compositional
sentence representations, any successful reasoning using compositional representations
will rely on the ability to perform sound inferences of this kind, so our first experiment studies how well each model can learn to perform them them in isolation.













\paragraph{Experiments}
We begin by creating a world model
on which we will base the statements in the train and test sets.
This takes the form of a small Boolean structure in which terms denote
sets of entities from a small domain.  Fig.~\ref{lattice-figure}a
depicts a structure of this form with three entities ($a$, $b$, and $c$) and eight proposition terms ($p_1$--$p_8$). We then generate a 
labelal statement for each pair of terms in the model, as shown in Fig.~\ref{lattice-figure}b. 
We divide these statements evenly into train and test sets, and delete the test set
 examples which cannot be proven from the train examples, for which there is not enough information for even an ideal system to choose a correct label.
In each experimental run, we create a model with 80 terms over a domain of 7 elements, yielding a training set of 3200 examples and a test set of 
2960 examples.

We trained models with both the NN and NTN comparison functions on these
data sets.\footnote{Since this task relies crucially on the learning of a pair of vectors, no simpler version of our model is a viable baseline.} %+%
In both cases, the models are implemented as
described in \S\ref{methods}, but since the items being compared
are single terms rather than full tree structures, the composition
layer is not used, and the two models are not recursive. We simply present
the models with the (randomly initialized) embedding vectors for each
of two terms, ensuring that the model has no information about the terms
being compared except for the labels between them that appear in training.


\begin{figure}[t]
  \centering
  \begin{subfigure}[t]{0.45\textwidth}
    \centering
    \newcommand{\labelednode}[4]{\put(#1,#2){\oval(1.5,1)}\put(#1,#2){\makebox(0,0){$\begin{array}{c}#3\\\{#4\}\end{array}$}}}
    \setlength{\unitlength}{1cm}\scalebox{0.8}{
    \begin{picture}(5,5.5)
      \labelednode{2.50}{5}{}{a,b,c}
      
      \put(0.75,4){\line(3,1){1.5}}
      \put(2.5,4){\line(0,1){0.5}}
      \put(4.25,4){\line(-3,1){1.5}}
      
      \labelednode{0.75}{3.5}{p_1,p_2}{a,b}
      \labelednode{2.50}{3.5}{p_3}{a,c}
      \labelednode{4.25}{3.5}{p_4}{b,c}
      
      \put(0.75,2.5){\line(0,1){0.5}}
      \put(0.75,2.5){\line(3,1){1.5}}
      
      \put(2.5,2.5){\line(-3,1){1.5}}
      \put(2.5,2.5){\line(3,1){1.5}}
      
      \put(4.25,2.5){\line(0,1){0.5}}
      \put(4.25,2.5){\line(-3,1){1.5}}
      

      \labelednode{0.75}{2}{p_5,p_6}{a}
      \labelednode{2.50}{2}{}{b}
      \labelednode{4.25}{2}{p_7,p_8}{c}
      
      \put(2.5,1){\line(-3,1){1.5}}
      \put(2.5,1){\line(0,1){0.5}}
      \put(2.5,1){\line(3,1){1.5}}
      
      \labelednode{2.5}{0.5}{}{}
    \end{picture}}
    \caption{Example boolean structure. The terms $p_1$--$p_8$ name the sets. Not all sets have names, and  some sets have multiple names, so that learning $\nateq$ is non-trivial.}
  \end{subfigure}
  \qquad\small
    \begin{subfigure}[t]{0.43\textwidth}
    \centering \vspace{0.4cm}
    \setlength{\tabcolsep}{12pt}
    \begin{tabular}[b]{c  c}
      \toprule
      Train & Test \\
      \midrule
      $p_1 \nateq p_2$              & $p_2 \natneg p_7$ \\
      $p_1 \natrev p_5$              & $p_2 \natrev p_5$ \\
      $p_4 \natrev p_8$              & \strikeout{$p_5 \nateq p_6$} \\
      $p_5 \natalt p_7$              & \strikeout{$p_7 \natfor p_4$} \\
      $p_7 \natneg p_1$           & $p_8 \natfor p_4$ \\

      \bottomrule
    \end{tabular}

    \caption{A few examples of atomic statements about the
      model.  Test statements that are not provable from the training data shown are
      crossed out.}
  \end{subfigure}  
  \caption{Small example structure and data for learning label composition.}
  \label{lattice-figure}
\end{figure} 

\begin{table}[tp]
  \centering\small
  \begin{tabular}{ l r@{ \ }r r@{ \ }r }
    \toprule
    ~&\multicolumn{2}{c}{Train} & \multicolumn{2}{c}{Test}\\
    \midrule
    $\natind$ only &53.8 &(10.5)    &53.8 &(10.5) \\
    15d NN &				99.8&	(99.0) &94.0&(87.0) \\
    15d NTN 				& \textbf{100} & \textbf{(100)} & \textbf{99.6} & \textbf{(95.5)}\\
    \bottomrule
  \end{tabular}
  
  
  \caption{Performance on the semantic label experiments. These results and all other results on artificial data are reported as mean accuracy scores over five runs followed by mean macroaveraged F1 scores in parentheses. The ``$\natind$ only'' entries reflect the frequency of the most frequent class.}
  \label{joinresultstable}
\end{table}

\paragraph{Results} 
The resuts (Table \ref{joinresultstable}) show that NTN is able to accurately encode the labels between the terms in the geometric labels between their vectors, 
and is able to then use that information to recover labels that 
are not overtly included in the training data. The NN also generalizes fairly well, 
but makes enough errors that it remains an open question whether 
it is capable of learning representations with these properties. 
It is not possible for us to rule out the possibility that different optimization techniques or
further hyperparameter tuning could lead an NN model to succeed here.

As an example from our test data, both models correctly labeled $p_1 \natfor p_3$, potentially learning from the training examples $\{p_1 \natfor p_{51},~p_3 \natrev p_{51}\}$ or $\{p_1\natfor p_{65},~p_3 \natrev p_{65} \}$. On another example involving comparably frequent labels, the NTN correctly labeled $p_6 \natrev p_{24}$, likely on the basis of the training examples $\{p_6 \natcov p_{28},~p_{28} \natneg p_{24}\}$, while the NN incorrectly assigned it $\natind$.


\section{Recursive structure}

\begin{figure}[t]
\begin{center}
\begin{tabular}{lll}
$a\equiv a$		&~~~&	$(c~(and~(not~d)))~\#~f$\\
$b~\#~c$			&~~~&	$(not~(c~(or~b)))~\sqsubset~(not~c)$\\
$d\natneg(not~d)$	&~~~&	$f~\#~((c~(or~(not~d)))~(and~a))$\\
$(c~(and~d))\sqsubset d$&~~~&$d\sqsupset((d~(or~d))~(and~(not~b)))$\\
\end{tabular}
\end{center}

\caption{Some sample randomly generated pairs of propositional logic statements.  \label{prop-figure}} 
\end{figure}

% TODO: Cite Chomsky/Hauser/Fitch?

Recursive structure is a prominent feature of natural language. Consider, for example, \ii{Alice said hello}, \ii{Bob said that Alice said hello}, and \ii{Carl thinks that Bob said that Alice said hello}. Overt recursion of this kind is easy to find, and theoretical accounts of natural language syntax and semantics rely heavily on recursive structures.
In order for a model to be able to accurately learn natural language meanings, then, we expect that it would need to be able to learn to represent the meanings of function words in a such a way that they are able to behave correctly when taking their own outputs as isput.

We again test this phenomenon within the framework of MacCartney and Manning-style entailment reasoning, but we replace the unanalyzed symbols from the previous experiment with expressions that involve recursive structure. To define these expressions, we turn to propositional logic, a relatively simple logic in which each variable represents either \ii{true} or \ii{false}. We generate data of the form seen in Figure \ref{prop-figure}: strings of arbitrary length consisting of six elementary proposition symbols and the operators \ii{and}, \ii{or}, and \ii{not}, arranged in pairs with the logical relations between them specified. It should be noted that we preserve the same  The data are generated randomly and deduplicated, and consist of about 248k training examples and 44k test examples. Each of the six symbols and each of the three operators is treated as a word for the purposes of our model, and is represented by a randomly initialized vector representation.

% TODO: Worth explicitly calling this project theorem proving?
Socher et al. \cite{socher2012semantic} have previously demonstrated the learning of a logic in a matrix-vector RNN model somewhat similar to our own, but the logic discussed here is substantially stronger, and a much better approximation of the kind of structure that is needed for natural language. The logic learned in that experiment is boolean, wherein the atomic symbols are simply the values 0 and 1, rather than variables over those values. While learning the operators of that logic is not trivial, the ouptuts of each operator can be represented accurately by a single bit. The statements of propositional logic learned here describe conditions on the truth values of propositions where those truth values are not known. As opposed to the two-way contrasts seen in \cite{socher2012semantic}, this logic distinguishes between 64 (2^6) possible assignments of truth values, and expressions of this logic define arbitrary conditions on these possible assignments, for a total of 2^{64} ($\approx 10^{20}$) possible statements that the recursive model needs to be able to distinguish.

...

% TODO: Add figure for results

train up to depth 5
test up to depth 12

% Describe data

\section{Reasoning with natural language quantifiers}

% Introduce the task

\subsection{Data}

Our data consists of pairs of sentences generated from a small artificial grammar. Each sentence contains a quantifier, a noun, which may be negated, and an intransitive verb which may be negated. We use the basic quantifiers \ii{some}, \ii{most}, \ii{all}, \ii{two}, and \ii{three}, and each of their duals over negation \ii{no}, \ii{not-all}, \ii{not-most}, \ii{less-than-two}, and \ii{less-than-three}. We also include five nouns, four intransitive verbs, and the negation symbol \ii{not}. In order to be able to define relations between sentences with differing lexical items, we define the lexical relations between each noun--noun pair, each verb--verb pair, and each quantifier--quantifier pair.

%nouns = ['warthogs', 'turtles', 'mammals', 'reptiles', 'pets']
%verbs = ['walk', 'move', 'swim', 'growl']
%dets = ['all', 'not_all', 'some', 'no', 'most', 'not_most', 'two', 'lt_two', 'three', 'lt_three']
%adverbs = ['', 'not']

To assign relation labels to sentence pairs, we built a small task-specific implemenation of MacCartney's logic that can accurately label sentences of this restricted language. The logic is not able to derive all intuitively true relations of this language, and fails to derive a single unique relation for certain types of statement, including De Morgean's laws (e.g. \ii{(all pets) growl $\natneg$ (some pet) (not growl)}), and we simply discard these examples. Exhaustively generating the valid sentences under this grammar and choosing those to which a relation label can be assigned
yields 66k sentence pairs. Some examples of these data are provided in Table \ref{examplesofdata}.

\begin{table}\small\centering
\begin{tabular}{|l|}
\hline
\ii{(most warthogs) walk $\natneg$ (not-most warthogs) walk}\\
\ii{(most mammals) move $\#$ (not-most (not turtles)) move}\\
\ii{(most (not pets)) (not swim) $\sqsupset$ (not-most (not pets)) move}\\
\hline
\ii{(no turtles) (not growl) $\|$ (no turtles) (not swim)}\\
\ii{(no warthogs) swim $\sqsupset$ (no warthogs) move}\\
\ii{(no warthogs) move $\sqsubset$ (no (not reptiles)) swim}\\
\hline
\end{tabular}
\caption{Sample data involving two different quantifier pairs.\label{examplesofdata}}
\end{table}

% TODO: Mention all seven relations seen, but some rare

\section{Experiments and results}

% TODO: Update with newer experiments

In the simplest experimental setting, which I label \textsc{all-split}, I randomly sample 85\% the data---making sure to sample 85\% of each of the individual datasets---train the model on that portion, and evaluate on the remaining 15\% of the data. This setting is meant to test whether the model is able to correctly generalize the individual reasoning patterns represented by each of the datasets. 

Performance on this setting is perfect: the model quickly converges to 100\% accuracy on the test data, showing that it is capable of accurately learning to capture all of the reasoning patterns in the data. The remaining experimental settings serve to determine whether what is learned captures the underlying logical structure of the data in a way that allows it to accurately label unseen kinds of reasoning pattern. In each of them, I choose one of three arbitrarily chosen target datasets, all involving quantifier substitution, to test on. I then then hold out that dataset and---depending on the experimental setting---other similar datasets from the training data in an attempt to discover just how different a test example can be from anything seen in training and still be classified accurately. Table \ref{patterntable} shows what information is included in the training data for each of the four settings for one of the three target datasets. 

\begin{table}\small\centering
\begin{tabular}{|l|l|}\hline
\textbf{Training configuration} & \textbf{Test accuracy}\\\hline
\textsc{all split}	&	TODO\% \\\hline
\textsc{pair no/no}	&	TODO\% \\
\textsc{pair two/less-than-two}	&	TODO\% \\
\textsc{pair not-all/not-most}	&	TODO\% \\
\textsc{pair all/some}	&	TODO\% \\\hline
\end{tabular}

\caption{Quantifier experiment performance.\label{resultstable}}
\end{table} % TODO: Replace

% TODO: Define experimental setups

\subsection{Discussion} 

% TODO: Revise

The model learns to generalize well to novel data in most but not all of the training configurations. This inconsistent performance suggests that there is room to improve the optimization techniques used on the model, but the fact that it is able to perform well in these settings even some of the time suggests that the structure of the model is basically capable of learning meaning representations that support inference.

The results in the \textsc{all-split} condition show two important behaviors. The model is able to learn to identify the difference between two unseen sentences and consistently return the label that corresponds to that difference. In addition, the model can learn lexical relations from the training data, such as \ii{dog $\sqsubset$ animal} from \ii{(no dog) bark $\sqsupset$ (no animal) bark}, and it can then use these learned lexical relations to compute the relation for a sentence pair like \ii{(some dog) bark $\sqsubset$ (all animal) bark}. The results from the other three experimental settings show that the model is able to learn general purpose representations for quantifiers which, at least in many cases, allow it to perform inference when a crucial difference between two sentences---the  substitution of one specific quantifier for another---has not been seen. These results serve to confirm that the representations learned are capturing some of the underlying logic, rather than than just supporting the memorization of fixed reasoning patterns. 

%  There has also been some research, most recently including \citet{grefenstette2013towards}, into models which represent quantifiers and other function words as higher order tensors rather than vectors, motivated in part by the hypothesis that their compositional behavior cannot be captured in strictly vector-based representatons. A successful inference engine that used those representations would not be a strict rebuttal of this assumption, but it could provide insight into precisely what circumstances would warrant the use of these richer tensor representation models in practice.  % TODO: Proof

These results leave open the question of how much information is minimally needed to learn general purpose representations for quantifiers in this setting. There are two lines of experimental work that could help to clarify this. Including longer sentences and constructions involving conjunctions (i.e. \ii{and, or}), transitive verbs (i.e. \ii{eats, kicks}) or other constructions in the training and test data could further test what kinds of behavior can be learned from a small training set, as could further experiments on this data involving even smaller training sets than those shown here or differently structured configurations of train and test sets.

%The representations that the model learns are not truly general-purpose, though. Human speakers of English who know the meanings of all of the words in a sentence are able to reason about whether that sentence entails any other sentence of English even if they have never seen either sentence before, or have never used the particular pattern of inference before. This broad ability to generalize does not appear to have been learned in these experiments. Pessimistically, the model might be primarily memorizing the reasoning patterns themselves, with any generalization beyond the \textsc{set-out} setting being erratic and case-specific. Optimistically, though, the model might be learning to handle logic in a way more like MacCartney's formal approach, in which the model must know dozens of different possible inference steps involving specific quantifiers and entailment patterns to succeed: if the model hasn't seen the substitution of \ii{most} with \ii{no}, it doesn't have the tools to reason about that substitution. 
%An incomplete ability to generalize is not necessarily detrimental to a model's ability to support reasoning in practice, especially in this later interpretation, since the body of knowledge used to do inference in MacCartney's approach is small enough that it could realistically be learned.

%It remains an open research question whether generalization of the type seen in the \textsc{pair-out} case is possible for any model. A model that could successfully generalize in this way could be potentially valuable both as a more versatile reasoning tool, and as a source of information about the ways in which functional types like quantifiers can be represented in vector space models.
%There are two ways that this problem can be addressed, one practical and one formal. 

% An obvious direction for future work is to test for this behavior in more powerful models, or to see whether a dataset containing a more diverse array of reasoning patterns (beyond those possible just with quantification and negation) could encourage generalization. Running experiments like the ones presented above on models with matrices or higher order tensors as word representations \cite{coecke2010mathematical, grefenstette2013towards} might be a promising start.
%It is also not impossible that even an RNTN like the one studied could generalize perfectly: the dramatic instability of the model over the course of training suggests that better approaches to learning could improve performance.

\section{The SICK textual entailment challenge}\label{sec:sick}



The specific model architecture that we use is novel, and though the underlying tree structure approach has been validated elsewhere, our experiments so far do not guarantee that it viable model for handling inference over real
natural language data. To investigate our models' ability to handle the noisy labels and the diverse range of linguistic structures seen in typical natural language data, we use the SICK textual entailment challenge corpus \cite{marelli2014sick}. The corpus consists of about 10k natural language sentence pairs, labeled with \ii{entailment}, \ii{contradiction}, or \ii{neutral}. At only a few thousand distinct sentences (many of them variants on an even smaller set of template sentences), the corpus is not large enough to train a high quality learned model of general natural language, but it is the largest human-labeled entailment corpus that we are aware of, and our results nonetheless show that tree-structured NN models can learn to do inference in the real world.

Adapting to this task requires us to make a few additions to the techniques discussed in \S\ref{methods}. In order to better handle rare words, we initialized our word embeddings using 200 dimensional vectors trained with 
GloVe \cite{pennington2014glove} on data from Wikipedia. Since 200 dimensional vectors are too large to be practical in an TreeRNTN on a small dataset, a new embedding transformation layer is needed. Before any embedding is used as an input to a recursive layer, it is passed through an additional $\tanh$ neural network layer with the same output dimension as the recursive layer. This new layer aggregates any usable information from the embedding vectors into a more compact working representation. An identical layer is added to the SumNN between the word vectors and the comparison layer.

We also supplemented the SICK training data\footnote{We tuned the model using performance on a held out development set, but report performance here for a version of the model trained on both the training and development data and tested on the 4,928 example SICK test set. We also report training accuracy on a small sample from each data source.} with 600k examples of entailment data from the Denotation Graph project (DG, \citealt{hodoshimage}, also used by the winning SICK submission), a corpus of noisy automatically labeled entailment examples over image captions, the same genre of text from which SICK was drawn. We trained a single model on data from both sources, but used a separate set of softmax parameters for classifying into the labels from each source. We parsed the data from both sources with the Stanford PCFG Parser v.~3.3.1 \cite{klein2003accurate}. We also found that we were able to train a working model much more quickly with an additional technique: we collapse subtrees that were identical across both sentences in a pair by replacing them with a single head word. The training and test data on which we report performance are collapsed in this way, and both collapsed and uncollapsed copies of the training data are used in training. Finally, in order to improve regularization on the noisier data, we used dropout \cite{srivastava2014dropout} at the input to the comparison layer (10\%) and at the output from the embedding transform layer (25\%). 

\begin{table}[tp]
  \centering \small
    \begin{tabular}{ l@{\hspace{-0.25em}} r@{~~~~} r@{~~~~} r@{~~~~} r@{~~~~} }
    \toprule
        ~&\ii{neutral}&	 30d  & 			30d & 50d\\
    ~&only &SumNN  &TrRNN &TrRNTN\\ 
     \midrule
    DG Train	& 50.0 & 68.0 & 67.0 & \textbf{74.0} \\
    SICK Train	& 56.7 & 96.6 & 95.4 & \textbf{97.8} \\
    SICK Test	& 56.7 & 73.4 & 74.9 & \textbf{76.9} \\
    \midrule
    \textsc{Passive} (4\%)	& 0 		& 76  		& 68		&\textbf{88}\\   
    \textsc{Neg} (7\%)		& 0 		& 96	 		& \textbf{100} & \textbf{100}\\
    \textsc{Subst} (24\%)	& 28 		& \textbf{72}  		& 64 		&  \textbf{72}\\
    \textsc{MultiEd} (39\%)	&  \textbf{68} & 61  		&66 		& 64 \\
    \textsc{Diff} (26\%)		& \textbf{96} &  	68		&79		& \textbf{96}\\  
    \midrule
    \textsc{Short} (47\%) & 50.0 & 73.9 & 73.5		& \textbf{77.3} \\    
    \bottomrule
  \end{tabular}
  \caption{Classification accuracy, including a category breakdown for SICK test data. Categories are shown with their frequencies.}
  \label{sresultstable}
\end{table} 

\begin{table*}[htp]
  \centering\small
  \begin{tabular}{l@{~~~}cl}
    \toprule
  The patient is being helped by the doctor	& \ii{entailment} & The doctor is helping the patient (\textsc{Passive})\\
    A little girl is playing the violin on a beach & \ii{contradiction} &	There is no girl playing the violin on a beach (\textsc{Neg})\\
    
    The yellow dog is drinking water from a bottle& \ii{contradiction} &	The yellow dog is drinking water from a pot  (\textsc{Subst})\\
        A woman is breaking two eggs in a bowl & \ii{neutral} &A man is mixing a few ingredients in a bowl (\textsc{MultiEd})\\
        Dough is being spread by a man & \ii{neutral} & A woman is slicing meat with a knife (\textsc{Diff})\\
    \bottomrule
  \end{tabular}
  \caption{\label{examplesofsickdata}Examples of each category used in error analysis from the SICK test data. }
\end{table*}


\paragraph{Results} Despite the small amount of high quality training data available and the lack of resources for learning lexical relationships, the results (Table~\ref{sresultstable}) show that our tree-structured models perform competitively on textual entailment, beating a strong baseline. Neither model reached the performance of the winning system (84.6\%), but the TreeRNTN did exceed that of eight out of 18 submitted systems, including several which used sophisticated hand-engineered features and lexical resources specific to the version of the entailment task at hand. 

To better understand our results, we manually annotated a fraction of the SICK test set, using mutually exclusive categories for passive/active alternation pairs (\textsc{Passive}), pairs differing only by the presence of negation (\textsc{Neg}), pairs differing by a single word or phrase substitution (\textsc{Subst}), pairs differing by multiple edits (\textsc{MultiEd}), and pairs with little or no content word overlap (\textsc{Diff}). Examples of each are in Table \ref{examplesofsickdata}. We annotated 100 random examples to judge the frequency of each category, and  continued selectively annotating until each category contained at least 25. We also use the category \textsc{Short} for pairs in which neither sentence contains more than ten words.
 
The results (Table \ref{examplesofsickdata}) show that the TreeRNTN performs especially strongly in the two categories which pick out specific syntactic configurations, \textsc{Passive} and \textsc{Neg}, suggesting that that model has learned to encode the relevant structures well. It also performs fairly on \textsc{Subst}, which most closely parallels the lexical entailment inferences addressed in \S\ref{sec:quantifiers}. In addition, none of the models perform dramatically better on the \textsc{Short} pairs than on the rest of the data, suggesting that the performance decay observed in \S\ref{sec:recursion} may not impact models trained on typical natural language text.

It is known that a model can perform well on SICK (like other natural language inference corpora) without taking advantage of compositional syntactic or semantic structure \cite{marelli2014semeval}, and our summing baseline model is powerful enough to do this. Our tree models nonetheless perform substantially better, and we remain confident that given sufficient data, it should be possible for the tree models, and not the summing model, to learn a truly high-quality solution.

\section{Results and discussion}\label{sec:discussion}

The results are shown in Figure \ref{prop-results}. All three models were able to largely memorize the training data (TODO, TODO, and TODO for $\le$6). With sufficient training data (in the $\le$6 setting), the two tree models were able to perform at above 99\% (TODO) on examples of size $\le$2 with a steady decay in performance continuing through TODO\% at size 12, and similar, if steeper, decays with smaller training sets. These robustly beat the simple baselines reported in Bowman et al.: the most frequent class occurs just over 50\% of the time, and a summing model that ignores word order does reasonably on the shortest examples but falls below 60\% by bin 4.

The LSTM performs fairly poorly in the $\le3$ setting, with performance at 3 a mediorce TODO\%, and an abrubt drop to TODO\% at 4, suggesting that the model did not acquire the needed generalizations. With more ample training data in the $\le6$ condition, though, the LSTM shows fairly good accuracy on short examples, and a smooth decay with no abrupt drop. 

\paragraph{Conclusion}

We find that all three \todo{Still using three?} models can exploit recursive structure to interpret sentences with complex unseen structures, and that tree structured model's biases allow them to learn to do this more effectively from less data. 
We interpret these results as evidence that the similar performance of tree and sequence models can be attributed to their fundamentally similar representational systems, with both models recognizing compositional structure when it is present, the former doing so more reliably, and the latter benefiting elsewhere---either because of architectural choices that make backpropagation more effective, or because interpreting sentences from left to right allows the model to better simulate other aspects of human language comprehension. We finally suggest that there is value in attempting to further develop this ability in sequence models without fully sacrificing the flexibility that makes these models succeed.


\subsubsection*{Acknowledgments}

We thank Jeffrey Pennington and Richard Socher, as well as Neha Nayak for developing the SICK collapsing technique.

We also gratefully acknowledge support from %
a Google Faculty Research Award, %
a gift from Bloomberg L.P., 
the Defense Advanced Research Projects Agency (DARPA) Deep Exploration and Filtering of Text (DEFT) Program under Air Force Research Laboratory (AFRL) contract no.~FA8750-13-2-0040,
the National Science Foundation under grant no.~IIS 1159679, and %
the Department of the Navy, Office of Naval Research, under grant no.~N00014-10-1-0109.
%
Any opinions, findings, and conclusions or recommendations expressed in this material are those of the authors and do not necessarily reflect the views of 
Google, 
Bloomberg L.P.,
DARPA,
AFRL
NSF, 
ONR, or 
the US government.

%\section{Introduction}
%
%
%The following instructions are directed to authors of papers accepted
%for publication in the NAACL HLT 2015 proceedings.  All authors are required
%to adhere to these specifications. Authors are required to provide 
%a Portable Document Format (PDF) version of
%their papers.  The proceedings will be printed on US-Letter paper.
%Authors from countries in which access to word-processing systems is
%limited should contact the publication chairs as soon as possible.
%
%\indent\paragraph{Note} Grayscale readability of all figures and
%graphics will be enforced for all accepted papers
%(\S\ref{ssec:accessibility}).  Apart from this, the style files and
%camera-ready requirements are unchanged from last year.
%
%\section{General Instructions}
%
%Manuscripts must be in two-column format.  Exceptions to the
%two-column format include the title, as well as the 
%authors' names and complete
%addresses (only in the final version, not in the version submitted for review), 
%which must be centered at the top of the first page (see
%the guidelines in Subsection~\ref{ssec:first}), and any full-width
%figures or tables.  Type single-spaced.  Do not number the pages.
%Start all pages directly under the top margin.  See the guidelines
%later regarding formatting the first page.
%
%%% If the paper is produced by a printer, make sure that the quality
%%% of the output is dark enough to photocopy well.  It may be necessary
%%% to have your laser printer adjusted for this purpose.  Papers that are too
%%% faint to reproduce well may not be included.
%
%%% {\bf Do not print page numbers on the manuscript.}  Write them lightly
%%% on the back of each page in the upper left corner along with the
%%% (first) author's name.
%
%The maximum length of a manuscript is eight (8) pages for the main
%conference, printed single-sided, plus two (2) pages for references
%(see Section~\ref{sec:length} for additional information on the
%maximum number of pages).  Do not number the pages.
%
%The review process is double-blind, so do not include any author information (names, addresses) when submitting a paper for review.  However, you should allocate space for the names and addresses so that they will fit in the final (accepted) version.  This is best done by either providing fake or blank names and addresses (as shown in this paper).
%
%\subsection{Electronically-available resources}
%
%NAACL HLT provides this description in \LaTeX2e{} ({\tt naaclhlt2015.tex}) and PDF
%format ({\tt naaclhlt2015.pdf}), along with the \LaTeX2e{} style file used to
%format it ({\tt naaclhlt2015.sty}) and an ACL bibliography style ({\tt naaclhlt2015.bst}).
%These files are all available at
%{\tt http://naacl2015.naacl.org}.  A Microsoft Word
%template file ({\tt naaclhlt2015.dot}) is also available at the same URL. We
%strongly recommend the use of these style files, which have been
%appropriately tailored for the NAACL HLT 2015 proceedings.
%
%
%\subsection{Format of Electronic Manuscript}
%\label{sect:pdf}
%
%For the production of the electronic manuscript you must use Adobe's
%Portable Document Format (PDF). This format can be generated from
%postscript files: on Unix systems, you can use {\tt ps2pdf} for this
%purpose; under Microsoft Windows, you can use Adobe's Distiller, or
%if you have cygwin installed, you can use {\tt dvipdf} or
%{\tt ps2pdf}.  Note 
%that some word processing programs generate PDF which may not include
%all the necessary fonts (esp. tree diagrams, symbols). When you print
%or create the PDF file, there is usually an option in your printer
%setup to include none, all or just non-standard fonts.  Please make
%sure that you select the option of including ALL the fonts.  {\em
%  Before sending it, test your {\/\em PDF} by printing it from a
%  computer different from the one where it was created}. Moreover,
%some word processor may generate very large postscript/PDF files,
%where each page is rendered as an image. Such images may reproduce
%poorly.  In this case, try alternative ways to obtain the postscript
%and/or PDF.  One way on some systems is to install a driver for a
%postscript printer, send your document to the printer specifying
%``Output to a file'', then convert the file to PDF.
%
%For reasons of uniformity, Adobe's {\bf Times Roman} font should be
%used. In \LaTeX2e{} this is accomplished by putting
%
%\begin{quote}
%\begin{verbatim}
%\usepackage{times}
%\usepackage{latexsym}
%\end{verbatim}
%\end{quote}
%in the preamble.
%`
%Additionally, it is of utmost importance to specify the {\bf
%  US-Letter format} (8.5in $\times$ 11in) when formatting the paper.
%When working with {\tt dvips}, for instance, one should specify {\tt
%  -t letter}.
%
%Print-outs of the PDF file on US-Letter paper should be identical to the
%hardcopy version.  If you cannot meet the above requirements about the
%production of your electronic submission, please contact the
%publication chairs above  as soon as possible.
%
%
%\subsection{Layout}
%\label{ssec:layout}
%
%Format manuscripts two columns to a page, in the manner these
%instructions are formatted. The exact dimensions for a page on US-letter
%paper are:
%`
%\begin{itemize}
%\item Left and right margins: 1 inch
%\item Top margin: 1 inch
%\item Bottom margin: 1 inch
%\item Column width: 3.15 inches
%\item Column height: 9 inches
%\item Gap between columns: 0.2 inches
%\end{itemize}
%	
%\noindent Papers should not be submitted on any other paper size. Exceptionally,
%authors for whom it is \emph{impossible} to format on US-Letter paper,
%may format for \emph{A4} paper. In this case, they should keep the \emph{top}
%and \emph{left} margins as given above, use the same column width,
%height and gap, and modify the bottom and right margins as necessary.
%Note that the text will no longer be centered.
%
%\subsection{The First Page}
%\label{ssec:first}
%
%Center the title, author's name(s) and affiliation(s) across both
%columns (or, in the case of initial submission, space for the names). 
%Do not use footnotes for affiliations.  Do not include the
%paper ID number assigned during the submission process. 
%Use the two-column format only when you begin the abstract.
%
%{\bf Title}: Place the title centered at the top of the first page, in
%a 15 point bold font.  (For a complete guide to font sizes and styles, see Table~\ref{font-table}.)
%Long title should be typed on two lines without
%a blank line intervening. Approximately, put the title at 1in from the
%top of the page, followed by a blank line, then the author's names(s),
%and the affiliation on the following line.  Do not use only initials
%for given names (middle initials are allowed). Do not format surnames
%in all capitals (e.g., ``Bangalore,'' not ``BANGALORE'').  The affiliation should
%contain the author's complete address, and if possible an electronic
%mail address. Leave about 0.75in between the affiliation and the body
%of the first page.
%
%{\bf Abstract}: Type the abstract at the beginning of the first
%column.  The width of the abstract text should be smaller than the
%width of the columns for the text in the body of the paper by about
%0.25in on each side.  Center the word {\bf Abstract} in a 12 point
%bold font above the body of the abstract. The abstract should be a
%concise summary of the general thesis and conclusions of the paper.
%It should be no longer than 200 words.  The abstract text should be in 10 point font.
%
%{\bf Text}: Begin typing the main body of the text immediately after
%the abstract, observing the two-column format as shown in 
%the present document.  Do not include page numbers.
%
%{\bf Indent} when starting a new paragraph. For reasons of uniformity,
%use Adobe's {\bf Times Roman} fonts, with 11 points for text and 
%subsection headings, 12 points for section headings and 15 points for
%the title.  If Times Roman is unavailable, use {\bf Computer Modern
%  Roman} (\LaTeX2e{}'s default; see section \ref{sect:pdf} above).
%Note that the latter is about 10\% less dense than Adobe's Times Roman
%font.
%
%\subsection{Sections}
%
%{\bf Headings}: Type and label section and subsection headings in the
%style shown on the present document.  Use numbered sections (Arabic
%numerals) in order to facilitate cross references. Number subsections
%with the section number and the subsection number separated by a dot,
%in Arabic numerals. 
%
%{\bf Citations}: Citations within the text appear
%in parentheses as~\cite{Gusfield:97} or, if the author's name appears in
%the text itself, as Gusfield~\shortcite{Gusfield:97}. In \LaTeX2e, the former is accomplished using
%\verb|\cite| and the latter with \verb|\shortcite| or \verb|\newcite|.
%Append lowercase letters to the year in cases of ambiguities.  
%Treat double authors as in~\cite{Aho:72}, but write as 
%in~\cite{Chandra:81} when more than two authors are involved. 
%Collapse multiple citations as in~\cite{Gusfield:97,Aho:72}.
%
%\textbf{References}: Gather the full set of references together under
%the heading {\bf References}; place the section before any Appendices,
%unless they contain references. Arrange the references alphabetically
%by first author, rather than by order of occurrence in the text.
%Provide as complete a citation as possible, using a consistent format,
%such as the one for {\em Computational Linguistics\/} or the one in the 
%{\em Publication Manual of the American 
%Psychological Association\/}~\cite{APA:83}.  Use of full names for
%authors rather than initials is preferred.  A list of abbreviations
%for common computer science journals can be found in the ACM 
%{\em Computing Reviews\/}~\cite{ACM:83}.
%
%The \LaTeX{} and Bib\TeX{} style files provided roughly fit the
%American Psychological Association format, allowing regular citations, 
%short citations and multiple citations as described above.
%
%{\bf Appendices}: Appendices, if any, directly follow the text and the
%references (but see above).  Letter them in sequence and provide an
%informative title: {\bf Appendix A. Title of Appendix}.
%
%\textbf{Acknowledgment} sections should go as a last (unnumbered) section immediately
%before the references.  
%
%\subsection{Footnotes}
%
%{\bf Footnotes}: Put footnotes at the bottom of the page. They may
%be numbered or referred to by asterisks or other
%symbols.\footnote{This is how a footnote should appear.} Footnotes
%should be separated from the text by a line.\footnote{Note the
%line separating the footnotes from the text.}  Footnotes should be in 9 point font.
%
%\subsection{Graphics}
%
%{\bf Illustrations}: Place figures, tables, and photographs in the
%paper near where they are first discussed, rather than at the end, if
%possible.  Wide illustrations may run across both columns and should be placed at
%the top of a page. Color illustrations are discouraged, unless you have verified that
%they will be understandable when printed in black ink. 
%
%\begin{table}
%\begin{center}
%\begin{tabular}{|l|rl|}
%\hline \bf Type of Text & \bf Font Size & \bf Style \\ \hline
%paper title & 15 pt & bold \\
%author names & 12 pt & bold \\
%author affiliation & 12 pt & \\
%the word ``Abstract'' & 12 pt & bold \\
%section titles & 12 pt & bold \\
%document text & 11 pt  &\\
%abstract text & 10 pt & \\
%captions & 10 pt & \\
%bibliography & 10 pt & \\
%footnotes & 9 pt & \\
%\hline
%\end{tabular}
%\end{center}
%\caption{\label{font-table} Font guide. }
%\end{table}
%
%{\bf Captions}: Provide a caption for every illustration; number each one
%sequentially in the form:  ``Figure 1. Caption of the Figure.'' ``Table 1.
%Caption of the Table.''  Type the captions of the figures and 
%tables below the body, using 10 point text.  
%
%\subsection{Accessibility}
%\label{ssec:accessibility}
%
%In an effort to accommodate the color-blind (as well as those printing
%to paper), grayscale readability for all accepted papers will be
%enforced.  Color is not forbidden, but authors should ensure that
%tables and figures do not rely solely on color to convey critical
%distinctions.
%
%\section{Length of Submission}
%\label{sec:length}
%
%The NAACL HLT 2015 main conference accepts submissions of long papers
%and short papers.  The maximum length of a long paper manuscript is
%eight (8) pages of content and two (2) additional pages of references
%\emph{only} (appendices count against the eight pages, not the
%additional two pages).  The maximum length of a short paper manuscript
%is four (4) pages and two (2) additional pages of references.
%Accepted papers will be granted an additional content page. For both
%long and short papers, all illustrations, references, and appendices
%must be accommodated within these page limits, observing the
%formatting instructions given in the present document.  Papers that do
%not conform to the specified length and formatting requirements are
%subject to be rejected without review.
%
%\section{Double-blind review process}
%\label{sec:blind}
%
%As the reviewing will be blind, the paper must not include the
%authors' names and affiliations.  Furthermore, self-references that
%reveal the author's identity, e.g., ``We previously showed (Smith,
%1991) ...'' must be avoided. Instead, use citations such as ``Smith
%previously showed (Smith, 1991) ...'' Papers that do not conform to
%these requirements will be rejected without review. In addition,
%please do not post your submissions on the web until after the
%review process is complete (in special cases this is permitted: see 
%the multiple submission policy below).
%
%We will reject without review any papers that do not follow the
%official style guidelines, anonymity conditions and page limits.
%
%\section{Multiple Submission Policy}
%
%Papers that have been or will be submitted to other meetings or
%publications must indicate this at submission time. Authors of
%papers accepted for presentation at NAACL HLT 2015 must notify the
%program chairs by the camera-ready deadline as to whether the paper
%will be presented. All accepted papers must be presented at the
%conference to appear in the proceedings. We will not accept for
%publication or presentation papers that overlap significantly in
%content or results with papers that will be (or have been) published
%elsewhere.
%
%Preprint servers such as arXiv.org and ACL-related workshops that
%do not have published proceedings in the ACL Anthology are not
%considered archival for purposes of submission. Authors must state
%in the online submission form the name of the workshop or preprint
%server and title of the non-archival version.  The submitted version
%should be suitably anonymized and not contain references to the
%prior non-archival version. Reviewers will be told: ``The author(s)
%have notified us that there exists a non-archival previous version
%of this paper with significantly overlapping text. We have approved
%submission under these circumstances, but to preserve the spirit
%of blind review, the current submission does not reference the
%non-archival version.'' Reviewers are free to do what they like with
%this information.
%
%Authors submitting more than one paper to NAACL HLT must ensure
%that submissions do not overlap significantly ($>25\%$) with each other
%in content or results. Authors should not submit short and long
%versions of papers with substantial overlap in their original
%contributions.

%%%

% \section*{Acknowledgments}

% Do not number the acknowledgment section.

\bibliographystyle{acl}
\bibliography{MLSemantics} 

\end{document}
