\documentclass{article} % For LaTeX2e
\usepackage{nips14submit_e,times}
\usepackage{hyperref}
\usepackage{url}
\usepackage[leqno, fleqn]{amsmath}
\usepackage{amssymb}
\usepackage{qtree}
\usepackage[numbers]{natbib}
\usepackage{graphicx}
\usepackage{booktabs}
\usepackage{colortbl}
\usepackage{caption}
\usepackage{subcaption}
\usepackage{xcolor}



\definecolor{mylinkcolor}{rgb}{0,0,0} % black
\hypersetup{colorlinks, linkcolor=mylinkcolor, urlcolor=mylinkcolor, citecolor=mylinkcolor}

\newcommand{\nateq}{\equiv}
\newcommand{\natind}{\mathbin{\#}}
%\newcommand{\natneg}{\raisebox{2px}{\tiny\thinspace$\wedge$\thinspace}}
\newcommand{\natneg}{\mathbin{^{\wedge}}}
\newcommand{\natfor}{\sqsubset}
\newcommand{\natrev}{\sqsupset}
\newcommand{\natalt}{\mathbin{|}}
\newcommand{\natcov}{\mathbin{\smallsmile}}

\newcommand{\plneg}{\mathop{\textit{not}}}
\newcommand{\pland}{\mathbin{\textit{and}}}
\newcommand{\plor}{\mathbin{\textit{or}}}



% Strikeout
\newlength{\howlong}\newcommand{\strikeout}[1]{\settowidth{\howlong}{#1}#1\unitlength0.5ex%
\begin{picture}(0,0)\put(0,1){\line(-1,0){\howlong\divide\unitlength}}\end{picture}}

\newcommand{\True}{\texttt{T}}
\newcommand{\False}{\texttt{F}}
\usepackage{stmaryrd}
\newcommand{\sem}[1]{\ensuremath{\llbracket#1\rrbracket}}


\renewcommand{\bibsection}{\subsubsection*{References}}

\usepackage{gb4e}

\def\ii#1{\textit{#1}}

\newcommand{\mynote}[1]{{\color{red}\framebox{\begin{tabular}{p{0.9\textwidth}}\footnotesize#1 \end{tabular}}}}


\title{Recursive Neural Networks for Learning Logical Semantics}

\author{
Samuel R.\ Bowman$^{\ast\dag}$ \\
\texttt{sbowman@stanford.edu} \\[2ex]
$^{\ast}$Stanford Linguistics \\
\And
Christopher Potts$^{\ast}$\\
\texttt{cgpotts@stanford.edu} \\[2ex]
$^{\dag}$Stanford NLP Group
\And
Christopher D.\ Manning$^{\ast\dag\ddag}$\\
\texttt{manning@stanford.edu}\\[2ex]
$^{\ddag}$Stanford Computer Science
}

% \author{
% Samuel R.\ Bowman \\
% NLP Group, Dept.\ of Linguistics\\
% Stanford University\\
% Stanford, CA 94305-2150 \\
% \texttt{sbowman@stanford.edu}
%  \And
%  Christopher Potts \\
% Dept.\ of Linguistics\\
% Stanford University\\
% Stanford, CA 94305-2150 \\
% \texttt{cgpotts@stanford.edu}
%  \And
% Christopher D.\ Manning \\
% NLP Group,  Depts.\ of Computer Science and Linguistics\\
% Stanford University\\
% Stanford, CA 94305-2150 \\
% \texttt{manning@stanford.edu}
% }

\newcommand{\fix}{\marginpar{FIX}}
\newcommand{\new}{\marginpar{NEW}}

\nipsfinalcopy % Uncomment for camera-ready version

\begin{document}


\begin{table}[tp]
  \centering
  \setlength{\tabcolsep}{10pt}
  \begin{tabular}{ l rr r r }
    \toprule
    Data & \multicolumn{1}{c}{Chance} & \multicolumn{1}{c}{16 dim RNN}  & \multicolumn{1}{c}{20 dim RNTN}\\
    \midrule
    \textsc{all split}              & 35.4  &  67.4& \textbf{100.0} 
    \\[1ex]    
    \textsc{two/less-than-two} & 59.8 & 77.2  &  \textbf{100.0}  \\
    \textsc{not-all/not-most}  &    0    & 66.0  &  \textbf{93.8}   \\
    \textsc{all/some}      &    0   & 62.7   & \textbf{78.4}  \\
    \textsc{no/no}             & 30.8  & 67.8  &  \textbf{99.9}   \\
    \bottomrule
  \end{tabular}
  \caption{Performance on the quantifier experiments. Results are reported as accuracy scores followed by macroaveraged F1 scores in parentheses.}
  \label{resultstable}
\end{table} 


\begin{figure}[hp]
  \centering\resizebox{4.5in}{!}{
  \footnotesize

\newcommand{\labeledtreenode}[4][3.5]{\put(#2){\makebox(0,0){{\fcolorbox{black}{#4}{\makebox(#1,0.3){#3}}}}}}

\newcommand{\textlabel}[4][3.5]{\put(#2){\makebox(0,0){{\fcolorbox{white}{white}{\makebox(#1,0.3){#3}}}}}}

\definecolor{lexcolor}{HTML}{F5F7C4}
\definecolor{compositioncolor}{HTML}{BBEBFF}
\definecolor{comparisoncolor}{HTML}{FFC895}
\definecolor{softmaxcolor}{HTML}{A5FF8A}


\setlength{\unitlength}{0.61cm}
\begin{picture}(21,7.5)
  
  \labeledtreenode[2.4]{11.5,7}{$P(\sqsubset) = 0.8$}{softmaxcolor}  
  \put(11.5,5.7){\vector(0,1){1}}  
  \labeledtreenode[7.85]{11.5,5.4}{all reptiles walk \emph{vs.}~some turtles move}{comparisoncolor}


  \textlabel{8,7}{Softmax classifier}{black}
  \textlabel{4.5,5.4}{Comparison N(T)N layer}{black}
      
  \textlabel{11.75,3.6}{Composition RN(T)N layers}{black}

  \textlabel{5,0.1}{Learned, randomly initialized word vectors}{black}
  
  %%%%%%%%%%%%%%%%%%%%%%%%%%%%%%%%%%%%%%%%%%%%%%%%%%
    
  \put(1.75,1.35){\vector(2,1){1.7}}
  \labeledtreenode{1.75,1}{all}{lexcolor}

  \put(6,1.35){\vector(-2,1){1.7}}
  \labeledtreenode{6,1}{reptiles}{lexcolor}

  \put(4,2.75){\vector(2,1){1.7}}
  \labeledtreenode{4,2.5}{all reptiles}{compositioncolor}

  \put(8.25,2.75){\vector(-2,1){1.7}}
  \labeledtreenode{8.25,2.5}{walk}{lexcolor}

  \put(6.25,4.25){\vector(4,1){3.25}}
  \labeledtreenode{6.25,3.9}{all reptiles walk}{compositioncolor}
  
  %%%%%%%%%%%%%%%%%%%%%%%%%%%%%%%%%%%%%%%%%%%%%%%%%%%

  \put(12.75,1.35){\vector(2,1){1.7}}
  \labeledtreenode{12.75,1}{some}{lexcolor}

  \put(17,1.35){\vector(-2,1){1.7}}
  \labeledtreenode{17,1}{turtles}{lexcolor}

  \put(15,2.75){\vector(2,1){1.7}}
  \labeledtreenode{15,2.5}{some turtles}{compositioncolor}

  \put(19.25,2.75){\vector(-2,1){1.7}}
  \labeledtreenode{19.25,2.5}{move}{lexcolor}
          
  \put(17.25,4.25){\vector(-4,1){3.25}}
  \labeledtreenode{17.25,3.9}{some turtles move}{compositioncolor}
  
\end{picture}


}
  % \caption{The model structure used to compare \ii{((all reptiles) walk)} and \ii{((some turtles) move)}. 
  %  The same structure is used for both the RNN and RNTN layer functions.} 
  \label{sample-figure}
\end{figure}

\end{document}